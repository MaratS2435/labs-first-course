\documentclass[10pt]{book}
\usepackage[a5paper,top=54pt,bottom=54pt,left=48pt,right=48pt]{geometry}
\usepackage[utf8]{inputenc}
\usepackage[T1,T2A]{fontenc}
\usepackage[english,russian]{babel}
\usepackage{graphicx}
\usepackage{amsmath,amsthm,amssymb}
\usepackage{caption2}
\usepackage{bm}


%page header
\usepackage{fancybox,fancyhdr}
\pagestyle{fancy}
\fancyhead{}
\fancyfoot{}
\renewcommand{\headrulewidth}{0pt}

%remove colon after "Рис. %number%"
\renewcommand{\captionlabeldelim}{~}

%font
\fontfamily{lh}
\selectfont

\usepackage{pgfpages}

\begin{document}


    \small{Если базис (как упорядоченный набор векторов) представить в виде символической матрицы-строки
    $(\textit{\textbf{e}})=(\textit{\textbf{e}}_1,\dots,\textit{\textbf{e}}_n)=(\textit{\textbf{e}}_{1}\;\dotsi\;\textit{\textbf{e}}_n)$ , то разложение вектора \textbf{\textit{\boldmath $\nu$}} по базису (\textit{\textbf{e}}) можно записать следующим образом:

    $$\bm{\nu}=\nu_{1}\textit{\textbf{e}}_{1}+\nu_{2}\textit{\textbf{e}}_{2}+\dotsc+\nu_{n}\textit{\textbf{e}}_{n}=(\textit{\textbf{e}}_{1}\;\dotsi\;\textit{\textbf{e}}_{n})
    \begin{pmatrix}
    \nu_1 \\
    \vdots \\
    \nu_n \\
    \end{pmatrix}\text{.}$$

    \noindent Здесь умножение символической матрицы-строки (\textit{\textbf{e}}) на числовую матрицу-столбец $\underset{(\textit{\textbf{e}})}{\nu}$ производится по правилам умножения матриц (см. разд.1 в [7]). Обозначение базиса у координатного столбца будем опускать, если понятно относительно какого базиса получены координаты\par

    \begin{center}
        \textbf{Линейные операции в координатной форме}
    \end{center}

    Пусть $\mathbf{e}_1,\mathbf{e}_2,\dots,\mathbf{e}_n$ --- базис пространства $\mathbf{V}$, векторы $\mathbf{u}$ и $\mathbf{\nu}$ имеют в этом базисе координаты $u=(u_{1}\;\dotsi\;u_{n})^\mathit{T}$ и $\nu=(\nu_{1}\;\dotsi\;\nu_{n})^\mathit{T}$ соответсвенно, т.е.$$\mathbf{u}=u_{1}\mathbf{e}_{1}+u_{2}\mathbf{e}_{2}+\dotsc+u_{n}\mathbf{e}_{n}\text{,}\qquad\mathbf{\nu}=\nu_{1}\mathbf{e}_{1}+\nu_{2}\mathbf{e}_{2}+\dotsc+\nu_{n}\mathbf{e}_{n}$$\par

    Из свойства 1 базиса следует, что \textit{равные векторы имеют равные соответствующие
    координаты (в одном и том же базисе)}, и наоборот, \textit{если координаты векторов (в одном и
    том же базисе) соответственно равны, то равны и сами векторы}.\par

    \textit{При сложении векторов их координаты складываются:}$$\mathbf{u}+\mathbf{\nu}=(u_{1}+\nu_{1})\mathbf{e}_1+(u_{2}+\nu_{2})\mathbf{e}_2+\dotsc+(u_{n}+\nu_{n})\mathbf{e}_n\text{.}$$
    \textit{При умножении вектора на число все его координаты умножаются на это число:}\par$$\lambda\mathbf{\nu}=(\lambda\nu_{1})\mathbf{e}_1+(\lambda\nu_{2})\mathbf{e}_2+\dotsc+(\lambda\nu_{n})\mathbf{e}_n\text{.}$$
    Другими словами, \textit{сумма векторов} $\mathbf{u}+\mathbf{\nu}$ \textit{имеет координаты} $u+\nu$, \textit{произведение} $\lambda\mathbf{\nu}$ \textit{имеет координаты} $\lambda\nu$. Разумеется, что все координаты получены в одном базисе $(\mathbf{e})=(\mathbf{e}_1,\dotsc,\mathbf{e}_n)$.\par
    Если система векторов линейно зависима (линейно независима), то их координатные
    столбцы, полученные относительно одного базиса, образуют линейно зависимую (соответственно, линейно независимую) систему.\par
    Все свойства линейной зависимости и линейной независимости векторов переносятся
    без изменений на их координатные столбцы, полученные в одном и том же базисе. И наоборот, свойства, сформулированные в [7] для матриц-столбцов, переносятся на векторы, если
    матрицы-столбцы считать их координатными столбцами.\vfill
    \begin{center}
        15
    \end{center}
    \pagebreak

    Пусть заданы два базиса пространства $\mathbf{V}\text{: }(\mathbf{e})=(\mathbf{e}_{1},\mathbf{e}_{2},\dotsc,\mathbf{e}_{n})$ и $(\mathbf{e'})=(\mathbf{e'}_{1},\mathbf{e'}_{2},\dotsc,\mathbf{e'}_{n})$. Базис $(\mathbf{e})$ будем условно называть «старым», а базис $(\mathbf{e'})$ - «новым».  Пусть известны разложения
    каждого вектора нового базиса по старому базису:\par
    $$\mathbf{e'}_i=s_{1i}\mathbf{e}_1+s_{2i}\mathbf{e}_2+\dotsc+s_{ni}\mathbf{e}_n\text{,}\qquad\text{i}=1,2,\dotsc,n$$

    Записывая по столбцам координаты векторов $\mathbf{e'}_1,\mathbf{e'}_2,\dotsc,\mathbf{e'}_n$ в базисе $\mathbf{(e)}$, можно составить матрицу:$$\underset{\mathbf{(e)\rightarrow(e')}}{S}=
    \begin{pmatrix}
        s_{11} & \cdots & s_{1n} \\
        \vdots & \ddots & \vdots \\
        s_{n1} & \cdots & s_{nn}
    \end{pmatrix}$$

    \noindent Квадратная матрица S , составленная из координатных столбцов векторов нового базиса ) $\mathbf{(e')}$
    в старом базисе $\mathbf{(e)}$, называется \textit{\textbf{матрицей перехода}} от старого базиса к новому. При помощи матрицы перехода (1.5) формулы (1.4) можно записать в виде: $$\mathbf{(e'_1\;\cdots\;e'_n)=(e_1\;\cdots\;e_n)}\cdot S\quad\text{  или, короче,}\;\mathbf{(e')=(e)}\cdot S$$
    \noindentУмножение символической матрицы-строки $\mathbf{(e)}$ на матрицу перехода S в (1.6) производится по правилам умножения матриц (см. раздю 1 в [7]).\par
    Пусть в базисе $\mathbf{(e)}$ вектор $\mathbf{\nu}$ имеет координаты $\nu_1,\nu_2,\dotsc,\nu_n$, а в базисе $\mathbf{(e')}$ - координаты $\nu'_1,\nu'_2,\dotsc,\nu'_n$, т.е.
    $$\mathbf{\nu}=\nu_1\mathbf{e}_1+\nu_2\mathbf{e}_2+\dotsc+\nu_n\mathbf{e}_n=\nu_1\mathbf{e}_1+\nu'_2\mathbf{e'}_2+\dotsc+\nu'_n\mathbf{e'}_n$$
    \noindentили, короче, $\mathbf{\nu}=\mathbf{(e)}\nu=\mathbf{(e')}\nu'$.\par
    \textit{Координатный столбец вектора в старом базисе получается в результате умножения
    матрицы перехода на координатный столбец вектора в новом базисе:}
    $$\underset{\mathbf{(e)}}{\nu}=S\underset{(e')}{\nu'}\qquad\text{или, что то же самое,}\quad
    \begin{pmatrix}
        \nu_1 \\
        \vdots \\
        \nu_n
    \end{pmatrix}=
    \begin{pmatrix}
        s_{11} & \cdots & s_{1n} \\
        \vdots & \ddots & \vdots \\
        s_{n1} & \cdots & s_{nn}
    \end{pmatrix}
    \begin{pmatrix}
        \nu'_1 \\
        \vdots \\
        \nu'_n
    \end{pmatrix}\text{.}$$
    \begin{center}
        \textbf{Свойства матрицы перехода от одного базиса к другому}
    \end{center}
    \textbf{1.}\textit{ Пусть имеются три базиса} $\mathbf{(e),(f),(g)}$ \textit{пространства \textbf{V} и известны матрицы перехода: }$\underset{\mathbf{(e)\rightarrow(f)}}{S}$ \textit{ от базиса }$\mathbf{(e)}$ \textit{ к базису } \textbf{(f)}; $\underset{\mathbf{(f)\rightarrow(g)}}{S}$ \textit{от} \textbf{(f)} \textit{к} \textbf{(g)};\linebreak $\underset{\mathbf{(e)\rightarrow(g)}}{S}$ \textit{от} \textbf{(e)} \textit{к} \textbf{(g)}. \textit{Тогда}
    $$\underset{\mathbf{(e)\rightarrow(f)}}{S}=\underset{\mathbf{(f)\rightarrow(g)}}{S}\underset{\mathbf{(f)\rightarrow(g)}}{S}$$\vfill
    \begin{center}
        16
    \end{center}
\end{document}
